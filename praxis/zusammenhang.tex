\subsection{Zusammenhang zwischen zwei Merkmalen}
%------------------------------------------------------------------------------------
\subsubsection{Regressionsanalyse}
\paragraph{Zusammehang zwischen Merkmalen}\label{praxis:regression:zusammenhang:1}
$n_w = 1500$ weibliche und $n_m = 500$ männliche Kunden haben alle einen Artikel gekauf, der in verschiedenen Farben verfügbar ist:
\begin{table}[ht]
\centering
\begin{tabular}{@{}lllll@{}}
\toprule
Farbe & Blau & Grün & Rot & Summe \\  \midrule 
Häufigkeit & 1000 & 600 & 400 & 2000 \\ \bottomrule
\end{tabular}
\end{table}
Wenn das Einkaufsverhalten unabhängig ist, so müssten die relativen Häufigkeiten bezogen auf Männer und Frauen gleich sein, oder mindestens sehr nahe beieinander liegen(siehe \autoref{eq:zusammenhang:1}).
\begin{table}[ht]
\centering
\begin{tabular}{@{}cllll@{}}
\toprule
\multirow{2}{*}{Geschlecht} & \multicolumn{4}{c}{Farbe} \\
 & Blau & Grün & Rot & Summe \\ \midrule
Weiblich & $(1000\cdot1500)\div2000$ & $(600\cdot1500)\div2000$ & $(400\cdot1500)\div2000$ & 1500 \\
Männlich & $(1000\cdot150)\div2000$ & $(1000\cdot150)\div2000$ & $(1000\cdot150)\div2000$ & 500 \\
\multicolumn{1}{l}{Summe} & \multicolumn{1}{l}{1000} & 600 & 400 & 2000 \\ \bottomrule
\end{tabular}
\end{table}\\
\smallskip
Wenn man nun die Resultate mit einer Tabelle aus, beispielsweise dem Vorjahr, vergleicht, und die Werte über "Pi-mal-Daumen" passen, sind die Merkmale eher voneinander abhängig.
\pagebreak
\paragraph{Minimiere die Summenformel}
\begin{equation*}
\sum_i^n(y_i - a - bx_i)^2
\end{equation*}
\begin{enumerate}
\item Partielles Ableiten des Ausdrucks nach $a$ und nach $b$
\item Nullsetzen der beiden partiellen Ableitungen
\item Auflösen der beiden Gleichungen nach $a$ und $b$
\end{enumerate}
\begin{equation*}
\ra{1.7}
\begin{array}{rcl}
a_1 &=& \overline{y}+b_1\cdot\overline{x}\\
b_1 &=& \displaystyle\frac{\sum_{i=1}^n x_i y_i - n\cdot\overline{x}\cdot\overline{y}}{\sum_{i=1}^n x_i^2 - n\overline{x}^2}
\end{array}
\end{equation*}
\subsubsection{Lineare Regression}
\paragraph{Beispiel} 12 Studenten gingen im letzten Semester neben dem Studium einer Erwerbstätigkeit nach. In der nachstehenden Tabelle sind für die 12 Studenten A bis L der zeitliche Aufwand (Std/Woche) für die Erwerbstätigkeit X und der zeitzliche Aufwand (Std./Woche) für das Studium Y angegeben
\smallskip
\begin{table}[H]
\ra{1.2}
\centering
\begin{tabular}{@{}lllllllllllll@{}}
\toprule
Student & A & B & C & D & E & F & G & H & I & J & K & L \\ \midrule
Erwerbstätigkeit & 1 & 2 & 2 & 3 & 3 & 4 & 5 & 6 & 8 & 12 & 15 & 33 \\
Studium & 39 & 37 & 36 & 40 & 36 & 37 & 34 & 36 & 33 & 33 & 32 & 27 \\ \bottomrule
\end{tabular}
\end{table}
Ein Student, der für die Bestreitung seines Lebensunterhalts 6 Stunden pro Woche erwärbstätig sein muss, will anhand der vorliegenden Daten ermitteln, wieviel Zeit er für sein Studium aufbringen kann. Die Berechnung erfolgt mit \autoref{eq:regression:4-1} und \autoref{eq:regression:4-2}.
\begin{table}[H]
\ra{1.2}
\centering
\begin{tabular}{@{}llllll@{}}
\toprule
Student & $x_i$ & $y_i$ & $x_i y_i$ & $x_i^2$ & $y_i^2$ \\ \midrule
A & 1 & 39 & 39 & 1 & 1521 \\
B & 2 & 37 & 74 & 4 & 1369 \\
C & 2 & 36 & 72 & 4 & 1296 \\
D & 3 & 40 & 120 & 9 & 1600 \\
E & 3 & 36 & 108 & 9 & 1296 \\
F & 4 & 37 & 148 & 16 & 1369 \\
G & 5 & 34 & 170 & 25 & 1156 \\
H & 6 & 36 & 216 & 36 & 1296 \\
I & 8 & 33 & 264 & 64 & 1089 \\
J & 12 & 33 & 396 & 144 & 1089 \\
K & 15 & 32 & 480 & 225 & 1024 \\
L & 23 & 27 & 621 & 529 & 729 \\
Summe & 84 & 420 & 2708 & 1066 & 14834 \\ \bottomrule
\end{tabular}
\end{table}
Wir können nun mit den oben erwähnten Gleichungen die Werte berechnen.
\begin{equation*}
\begin{array}{rcl}
b_1 &=& \frac{2708-12\cdot \frac{84}{12}\cdot \frac{420}{12}}{1066-12\cdot\left(\frac{84}{12}\right)^2} \\
&=& -\frac{116}{239} \\
&=& -0.49\\
a_1 &=&\frac{420}{12}-(-\frac{116}{239})\cdot \frac{84}{12} \\
&=& \frac{9177}{239} \\
&=& 38.43 
\end{array}
\end{equation*}
Daraus folgt zu allgemeinen Berechnung
\begin{equation*}
y=38.43 - 0.49 \cdot x \text{ mit } y(6) = 35.5\text{h}
\end{equation*}