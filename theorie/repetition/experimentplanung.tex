\subsection{Experimentplanung}
Ein Experiment besteht aus einem Input und Output. Man versucht dann einen Zusammenhang zwischen diesen beiden Mengen herzustellen. \\

Wir unterscheiden zwischen systematischen, zufälligen und absolutem Fehler. 
\subsubsection{Methoden zur Fehlerrechnung}
\begin{itemize}
\item Bei Summen und Differenzen, addieren sich die absoluten Fehler. Durch Genauigkeit, bzw Ungenauigkeit, Tolleranzen definieren. 
\item Bei Produkten und Quotienten, addieren sich die relativen Fehler.\\
Der absolute Fehler hat eine Masseinheit (diesselbe wie der Messwert). Der relative Fehler ist einheintenlos, üblich ist die Angabe in Prozent.
\item Da in beiden (absoluter und relativer Fehler) letztendlich genau die gleiche Information steckt, stellt sich die Frage, welcher Art der Fehlerangabe die bessere ist.
% TODO: Beispiele auf Folie 21, 22, 34
\end{itemize}
\subsection{Simulation als experimentelles Verfahren}
In der SImulation liegt ein experimentelles Verfahren vor. Um die Simulation erfolgreich durchzuführen:
\begin{itemize}
\item Muss ein Modell erstellt werden, welches das Realsystem in den interessierenden Fragen hinreichend genau abbildet (isolierende Abstraktion). Das Modell besteht aus dem System- und Verhaltensmodell hinsichtlich logischem und temporalem Verhalten. Dem Datenmodell der Eingangsdaten, der Störgrössen und der Streugrössen.
\end{itemize}
\subsection{Kontrollfragen}
\begin{itemize}
\item Was verstehen sie unter einer Hypothese?
\item Nach der Datenerhebung, bereitet man die Daten auf. Die Formel $n\cdot p \left(\frac{1}{p}\right)$. Bei einer Hypothese können zwei Fehler auftrete: Fehler der ersten und der zweiten Art. Der Fehler der ersten Art, hat man eine Pseudo-Gauss Verteilung, legt die Fehlertoleranzen falsch, und verwirft die Hyptohese fälschlicherweise (Produzentenrisiko). Bei dem Konsumentenrisiko,  haben wir die Hypothese überprüft und bestätigt, sie ist aber trotzdem falsch.
\end{itemize}