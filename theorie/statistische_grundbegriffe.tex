\subsection{Statistische Grundbegriffe}
\subsubsection{Merkmalsträger und Grundgesamtheit}
\begin{itemize}
\item Merkmalsträger\\
Der Merkmalsträger ist der Gegenstand der statistischen Untersuchung, er ist der Träger der interessierenden statistischen Information.
\item Grundgesamtheit\\
Die Grundgesamtheit ist die Menger aller Merkmalsträger. Die übereinstimmende Abgrenzungsmerkmale besitzen. Abgrenzungsmerkmale sind räumlich, sachlich und zeitlich vorzunehmen.
% TODO: Beispiele Folie 3
\item Merkmal\\
Die Eigenschaft des Merkmalsträger, die bei der statistischen Untersuchung von Interesse ist.
\item Merkmalswert\\
Der Wert, der bei der Beobachtung, der Befragung, der Messung oder Zählvorgang festgestellt wurde.
\end{itemize}
\subsubsection{Skalen}
\begin{tcolorbox}[colback=green!5,colframe=green!40!black,title=Skalen]
Die statistische Messskala, kurz Skala, ist dabei das Instrument, mit dem die Merkmalswerte ermittelt werden. SKala sind die möglichen Merkmalswerte nach einem bestimmten Ordnungsprinzip als Skalenwerte abtragen.
\end{tcolorbox}
\begin{itemize}
\item Nominalskala
\subitem Aus der Nominalskala sind die Skalenwerte Namen abgetragen die gleichberechtigt bzw. gleichbedeutend nebeneinander angeordnet sind. Sie sind stets qualitiative Merkmale
\begin{table}[ht]
\centering
\begin{tabular}{@{}ll@{}}
\toprule
Merkmal & Merkmalswert \\ \midrule
Familienstand & verheiratet, ledig, geschieden, verwitwert \\
Geschlecht & Feminin/Maskulin \\
Rebsorte & Riesling, Silvaner \\ \bottomrule
\end{tabular}
\end{table}
\item Ordinalskala
\subitem Auf der Ordnialskala (Rangskala) sind als Skalenwerte Klassenbeziehungen abgetragen. Die Skalenwerte stehen jetzt nicht mehr gleichberechtigt bzw. gleichwertig nebeneinander. Sondern sind entsprechende ihrer Klasse in auf- oder absteigender Folge (Rangfolge, Rangordnung) auf der Skala angeordnet. Ordinalskalierte Merkmale sind stets intensitätsmässig abgestufte Merkmale und umgekehrt
\begin{table}[ht]
\centering
\begin{tabular}{@{}ll@{}}
\toprule
Merkmal & Merkmalswert \\ \midrule
Schulnote & Sehr gut, gut, befriedigend, ausreichend, mangelhaft \\
Qualitätstufe & Standard, Business, First Class \\ \bottomrule
\end{tabular}
\end{table}
\item Metrische Skala
\subitem Auf er metrischen Skala (Kardinalskala) sind die Skalenwerte als reele Zahlen abgetragen, entsprechende ihrem Zahlenwert in auf- oder absteigender Folge auf der Skala angeordnet. Sie entspricht unserer Vorstellung von einem Experiment, wo als Ergebniss dem Merkmal eines Merkmalsträger als Merkmalwert ein eine reele Zahl zugewiesen wird. Metrische Merkmale sind stets quantitative Merkmale und umgekhert. Diese Skala wird abhängig vom Nullpunkt in Intervallskala und Verhältnisskala unterschieden.
\item Intervallskala
\subitem Auf der Intervallskala ist der Skalenwert Null ein mehr oder weniger willkürlich gewählter Nullpunkt. Das heisst, zwischen zwei Merkmalswerten kann der einfache Abstand (Intervall) gemessen werden. Es kann jedoch nicht der verhältnissmässige (relative) Abstand (Verhältniss, Quotient) gemessen werden.
\begin{table}[ht]
\centering
\begin{tabular}{@{}ll@{}}
\toprule
Merkmal & Merkmalswert \\ \midrule
Temperatur & $-12, \ldots, 0, \ldots, 42$ \\
Uhrzeit & 20:00, 0:00, 10:00 \\ \bottomrule
\end{tabular}
\end{table}
\item Verhältnisskala
\subitem
Auf der Verhältnisskala entspricht der Skalenwert Null dem natürlichen, absoluten Nullpunkt. Negative Werte sind damit nicht möglich. Das hat zur Folge, dass zwischen zwei Merkmalswerten neben dem einfachen Abstand (Intervall) auch der verhältnismässige Abstand (Quotient, Verhältnis) gemessen werden kann. Das heisst ein Merkmalswert kann jetzt als das Vielfache eines anderen Merkmalwertes ausgedrückt werden
\begin{table}[ht]
\centering
\begin{tabular}{@{}ll@{}}
\toprule
Merkmal & Merkmalswert \\ \midrule
Gewicht(kg) & 0,...10,...20,...40 \\
Alter(Jahre) & 0, 1,..40,...89,... \\ \bottomrule
\end{tabular}
\end{table}
\end{itemize}
\subsubsection{Bedeutung der Messkalen}
\begin{tcolorbox}[colback=green!5,colframe=green!40!black, title=Bedeutung der Messskalen]
Die Verhältnisskala besitzt das höchste Informationsniveau. Es lassen sich die Verschiedenartgkeit die einfachen und die verhältnissmässigen Abstände für Merkmalswerte feststellen.
\end{tcolorbox}
