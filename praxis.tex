%TODO: Beispiel hinzufügen, mit Formel verknüpfen
\subsection{Mittelwerte und Spannweite}\label{praxis:mittelwerte}
\begin{table}[h]
\ra{1.2}
\centering
\begin{tabular}{@{}lllllll@{}}
\toprule
Versicherungssumme\\ in Tsd & Anzahl Verträge & Median der Klasse & $x_ih_i$ & $H_j$ & Std. Abweichung & Varianz \\ \midrule
4-10 & 20 & 7 & 140 & 20 & 481 & 11568.05 \\
10-20 & 160 & 15 & 2400 & 180 & 2568 & 41214.4 \\
20-30 & 80 & 25 & 2000 & 260 & 484 & 2928.20 \\
30-40 & 40 & 35 & 1500 & 300 & 158 & 624.1 \\
40-80 & 88 & 50 & 5280 & 388 & 2547.6 & 73752.02 \\
80-120 & 12 & 100 & 1200 & 400 & 827.4 & 57049.23 \\ \bottomrule
\end{tabular}
\end{table}
\begin{itemize}
\item Durchschnittliche Versicherungssume(\autoref{theorie:mittelwerte:arith}):
\begin{equation*}
\overline{x} = \frac{1}{n}\sum_j x_jh_j = \frac{1}{400}\cdot 12420=3105
\end{equation*}
\item Dichte (\autoref{theorie:mittelwerte:dicht})
\begin{align*}
d_1=\frac{20}{10-4}=3.33, d_2=16, d_3=8, d_4=4, d_5=2.2, d_6=0.3
\end{align*}
\item Modus\\
Aus den Dichten lässt sich herauslesen, dass Klasse 2 die höchste Dichte hat, diese ist also die Modusklasse. Berechnung des Modus(\autoref{theorie:mittelwerte:modus:2})
\begin{equation*}
\ra{2}
\begin{array}{rcccl}
M_0 &=& x_2^u+\displaystyle\frac{d_2-d_1}{(d_1-d_1)+(d_2-d_3)}\cdot(x_2^o-x_2^u)&& \\
M_0 &=& 10+\displaystyle\frac{16-3.33}{(16-3.33)+(16-8)}\cdot(20-10) &=& 16130
\end{array}
\end{equation*}
\item Median\\
Der Median ist der Wert der geordneten Mitte von 400 (Summe aller Verträge), also 200. Diese Anzahl kommt in der 3. Klasse zu liegen. Damit ergibt sich der Median(\autoref{theorie:mittelwerte:quant}):
\begin{equation*}
\ra{2}
\begin{array}{rcccl}
M_e&=&x_3^u+\displaystyle\frac{\frac{n}{2}-H_2}{h_3}(x_3^o-x_3^u)&&\\
M_e&=&20+\displaystyle\frac{400\div2-180}{80}(30-20)&=&22500
\end{array}
\end{equation*}
\item Quantil\\
Die erste Quantilklasse it $\frac{400}{4}=100$. Dies ist in Klasse 2. Analog zum Median berechnet sich das erste Quantil(\autoref{theorie:mittelwerte:quant}).
\begin{equation*}
\begin{array}{rcccl}
Q_1&=&x_2^u+\displaystyle\frac{\frac{n}{4}-H_1}{h_2}(x_2^o-x_2^U)&&\\
Q_1&=&10\displaystyle\frac{400\div4-20}{160}(20-10)&=&50000
\end{array}
\end{equation*}
Das zweite Quantil ist der Median und das dritte Quantil entspricht dem ersten Quantil plus dem Median. 
\pagebreak
\item Quantils und Dezilabstände\\
\subitem Quantilsabstände: Der Quantilsabstand bewegt sich 50\% um den Median.
\begin{equation*}
\begin{array}{rcl}
Q_3-Q_1&=&ZQA\\
40000 - 15000&=&25000
\end{array}
\end{equation*}
\subitem Dezilabstände\\
\begin{equation*}
\begin{array}{rcccl}
D_j&=&x_j^u+\displaystyle\frac{n\cdot\frac{1}{10}-H_{j-1}}{h_j}\cdot(x_j^o - x_j^u)&&\\
D_9&=&40+\displaystyle\frac{400\cdot\frac{9}{10}-300}{88}\cdot(80-40)&=&67270
\end{array}
\end{equation*}
Der Dezilabstand ist $D_9-D_1=56020$.
\end{itemize}
%------------------------------------------------------------------------------------------
\subsection{Grundlagen der schliessenden Statistik}
\begin{enumerate}
\item Es sei X die Zufallsvariable X = Augenzahl bei Würfeln, dann galt für einen Wurf: $\mu=3.5, \sigma^2=2,92$. Wie sieht die Häufigkeit (Wahrscheinlichkeit), Varianz, Mittelwerte bei einem, zwei oder drei Würfen aus?\\
% TODO: Tabelle schreiben
Die 5 tritt 4 mal auf, die 6 Tritt 10 mal auf.
\begin{table}[ht]
\centering
\begin{tabular}{@{}llll@{}}
\toprule
 & 1. Wurf & 2. Wurf & 3. Wurf \\ \midrule
Mittelwert $\mu$ & 3.5 & 3.5 & 3.5 \\
Varianz$\sigma^2$ & 2.92 & $\frac{2.92}{2}=1.46$ & $\frac{2.92}{3}=0.97$ \\
Std. Abweichung $\sigma$ & 1.71 & 1.21 & 0.98 \\ \bottomrule
\end{tabular}
\end{table}
% TODO: Bilder der Würfe
\item Sie haben ein Situationsmodell einer Produktionseinrichtung erstellt, vom realen System wissen sie, dass diese Maschine im normalverteilten Mittel 10 Stück pro Sekunde produziert und die Standardabweichung von 1 Stück pro Sekunde besitzt. Sie führen der Simulationsumgebung 25 Experimente (Replications) durch. 
\subitem Mit welcher Wahrscheinlichkeit wird der mittlere Ausstoss zwischen 9.8 und 10.2 Stück pro Sekunde liegen?
\subitem Mit welcher Wahrscheinlichkeit wird der Ausstoss über 10.2 liegen?
\subitem Wie verändert sich das Ergebniss, wenn sie statt dessen nur 5 bzw. 100 Experimente machen?\\
Mit $\mu = \frac{10 S}{sec}$ und $\sigma=\frac{1 S}{sec}$. Daraus folgt:
\begin{equation*}
Z_{\mbox{Stichprobe}}=\frac{\overline{X}-\mu}{\frac{\sigma}{\sqrt{n}}}
\end{equation*}
Gesucht ist nun:
\begin{equation*}
\ra{2}
\begin{array}{rcl}
&P&(9.8\geq\overline{x}\geq10.2)\\
&P&\left(\displaystyle\frac{9.8-\mu}{\frac{\sigma}{\sqrt{n}}}\geq\displaystyle\frac{\overline{x}-\mu}{\frac{\sigma}{\sqrt{n}}}\geq\displaystyle\frac{10.2-\mu}{\frac{\sigma}{\sqrt{n}}}\right)\\
\Longrightarrow&P&\left(\displaystyle\frac{9.8-10}{0.2}\geq\displaystyle\frac{\overline{x}-10}{0.2}\geq\displaystyle\frac{10.2-10}{0.1}\right)\\
&P&(Z_1\geq\overline{Z}\geq Z_2)=1-\alpha
\end{array}
\end{equation*}
\pagebreak[4]
\item In welchem symmetrischen Intervall liegt der Mittelwert bei vorgegebener Wahrscheinlichkeit, oft mit $1-\alpha$ bezeichnet. Hier sei $1-\alpha=0.9$. 
\subitem Z ist nach Voraussetzung N(0,1) verteilt, dann folgt unmittelbar: $P(-z\leq\overline{z}\leq z)=1-\alpha \Rightarrow 1-\alpha = F_0(z)-F_0(-z)$, wobei z die obere und untere Schranke bildet: 
\begin{table}[ht]
\centering
\begin{tabular}{@{}llll@{}}
\toprule
$n$ & $\sigma\div n^2$ & $Z\cdot \sigma \div n^2$ & Intervall \\ \midrule
5 & 0.45 & 0.738 & [9.262;10.738] \\
25 & 0.2 & 0.328 & [9.672;10.328] \\
100 & 0.1 & 0.164 & [9.836;10.164] \\ \bottomrule
\end{tabular}
\end{table}
Dies finden wir wiederum mit der folgenden Formel heraus:
\begin{equation*}
\overline{Z}=\frac{\overline{X}-\mu}{\frac{\sigma}{\sqrt{n}}}
\end{equation*}
Daraus folgt \autoref{eq:stichprobenumfang:1}.
\end{enumerate}
%------------------------------------------------------------------------------------------
\subsection{Schätzverfahren}
\paragraph{Beispiel} In einer Wurstfabrik werden u.a. Leberwürste hergestellt. Aus langjährigen Messreihen ist bekannt, dass das Füllgewicht der Leberwürste normalvertwilt ist. Das Soll-Mindestgewicht der Würste beträgt 125g. Aus der Tagesproduktion von 600 Würsten wurder 26 Würste zufällig ohne Zrücklegen entnommen und gewogen. Die Messergebnisse für das Füllgewicht (in Gramm) wurden in eine Tabelle eingetragen.
\begin{table}[ht]
\centering
\begin{tabular}{@{}lllllll@{}}
\toprule
128.4 & 123.8 & 123.5 & 126.9 & 125.5 & 123.1 & 124.9 \\
123.1 & 126.6 & 121.9 & 125.3 & 123.4 & 122.1 & 124   \\
123.3 & 123.2 & 123.2 & 124   & 122.8 & 127.1 & 125.7 \\
127.1 & 125.8 & 123.7 & 125.9 & 124.9 &       &       \\ \bottomrule
\end{tabular}
\end{table}\\
Die Daten haben den Mittelwert: $ \overline{X}=124.5\text{g}$ und eine Varianz $s=1.72\text{g}$.
\subparagraph{Erstellen des 95\% Konfidenzintervall:} $\mu = 26$. Da $s = \sigma^2$:
\begin{align*}
z = \frac{\sqrt{1.72}}{\sqrt{26}} = 0.2572 \\
\end{align*}
Nun muss man den gesuchrten Wert in der Tabelle der Normalverteilung nachschlagen, in unserem Fall 0.975 (weil es das Mittel von 95\% ist, und die Verteilung Normalverteilt ist, können wir einfach die hälfte von 5 dazuzählen). Dabei erhalten wir 1.96. Damit nun das Intervall festgelegt werden kann, muss man erst z mit dem Wert der Tabelle multiplizieren, und dann das Intervall dazu bzw. abziehen.
\begin{align*}
0.272 \cdot 1.96 = 0.5041 \\
W(123.9959\leq124.5\leq125.0041)
\end{align*}
\subparagraph{Ermittlung der Konfidenz für das mit 125g nach unten begrenze Intervall für $\mu$}
Wir kennen aus \autoref{eq:stichprobenumfang:1} dass: $ \overline{\mu}-z\frac{\sigma}{\sqrt{n}} > 125\text{g}$. Da wir $n=26, \sigma^2=1.72, \overline{N}=124.5$ bereits kennen, können wir diese lediglich einsetzen.
\begin{equation*}
\begin{array}{rrcl}
&124.5-z\frac{\sqrt{1.72}}{\sqrt{26}}&>&125 \\
\Rightarrow&z&>&1.9440 \\
\rightarrow&z(97.5\%) && \
\end{array}
\end{equation*}
Den Wert für $z(x)$ muss man in der Tabelle "rückwärts" nachschlagen, also den entsprechenden Wert so genau wie möglich bestimmen und die Zahlen am linken und oberern Rand kombinieren.
%-----------------------------------------------------------------------------------------------------------------------------------------------
%TODO
%\paragraph{Beispiel} Ein Chemieunternehmen möchte den Bekanntheitsgrad eines von ihm hergetellten Waschmittels in Erfahrung bringen. Dazu werden 400 Personen zufällig ausgweählt und befragt. Das Waschmittel war bei 30\% der Befragten bekannt.
%\subparagraph{Erstellen eines zentralen 95\%-Konfidenzintervall für $\Theta$}
%TODO Folie 15 Schätzverfahren ausrechnen
%-----------------------------------------------------------------------------------------------------------------------------------------------
\pagebreak
\paragraph{Beispiel} Einer Lieferung von 1000 Paketen Zucker ist mit einer 95\% Konfidenz bei einem Fehler von $e=0.2\text{g}$ zu untersuchen, ob bei einer bekannten Standardabweichung von 1.2g der garantierte Mittelwert eingehalten wird. Wie viele Proben sind aus der Lieferung mindestens zu entnehmen?
\begin{itemize}
\item $Z$ aus Tabelle der Standardnormalverteilung: 1.96
\item Aus \autoref{eq:schätzverfahren:2} folgt: 
\begin{equation*}
\begin{array}{rcl}
 n&\geq&\frac{1.96^2\cdot1000\cdot1.2^2}{0.2^2\cdot(1000-1)+1.96^2\cdot1.2^2}  \\
 n&\geq&121.6 
\end{array}
\end{equation*}
\end{itemize}
\subsection{Testverfahren}
\subsubsection{Parametertest}
Eine Autozeitschrift möchte wissen, ob der Benzinverbrauch eines bestimmten Autotyps $\mu=\frac{10l}{100km}$ und mit $\sigma=\frac{1l}{100km}$ eingehalten wird. Dazu sollen 25 Autos untersucht werden.\\
Als Nullhypothese $H_0$ geht man davon aus, dass die Angabe stimmt.\\
Bestimmen des Stichprobenmittels der 25 Testfahrzeuge:
\begin{itemize}
\item einmal $\frac{10.2l}{100km}$ 
\item einmal $\frac{10.4l}{100km}$
\end{itemize}
Die Frage ist: 
\begin{itemize}
\item liegt iene systembedingte Störung und der angegebene Mittelwert ist falsch oder
\item liegt das Ergebnis im Bereich der statistischen Schwankungen (Toleranz)
\item Die Frage kann über das Vertrauensintervall beantwortet werden (\autoref{eq:parameter:1})
\end{itemize}
Mit den gegebenen Werten füllen wir die \autoref{eq:parameter:1} aus: $\mu_{0}=10, \sigma=1, n=25, 1-\alpha=0.95$
\begin{equation*}
10-1.96\frac{1}{\sqrt{25}}\leq\overline{x}\leq10+1.96\frac{1}{\sqrt{25}}
\end{equation*}
Unter der gegebenen Wahrscheinlichkeitsangabe würde die Nullhypothese mit dem Stichprobenmitel von 10.2 noch angenommen und der Unterschied als statistische Schwankung akzeptiert. Hingegen wird das Stichprobenmittel von 10.4 als signifikante Abweichung betrachten und somit die Nullhypothese zurückgewiesen. Der mögliche Fehler liegt bei $2.5\%$.
\subsection{Unabhängigkeitstest}
Ein Unternehmen stellt ein Produkt in vier Zweigwerken her. Je nach erreichter Qualität wird ein Produkt in eine von 3 Qualitätstufen eingeordnet. Es soll mittels einer Stichprobe von 500 Stück untersucht werde, ob zwischen der Qualität der Produkte und den Zweigwerken, in denen sie hergestellt werden, ein Zusammenhang beziehungsweise Abhängigkeit besteht\\
\begin{table}[htb]
\centering
\begin{tabular}{@{}lllll@{}} \toprule
 & \multicolumn{4}{l}{Qualitätsstufe} \\
\multirow{-2}{*}{Zweigwerk} & 1 & 2 & 3 & Summe \\ \midrule
1 & 125 & 25 & 10 & \cellcolor[HTML]{FFFC9E}160 \\
2 & 75 & 20 & 10 & \cellcolor[HTML]{FFFC9E}105 \\
3 & 60 & 25 & 15 & \cellcolor[HTML]{FFFC9E}100 \\
4 & 75 & 40 & 20 & \cellcolor[HTML]{FFFC9E}135 \\
Summe & \cellcolor[HTML]{FFFC9E}335 & \cellcolor[HTML]{FFFC9E}110 & \cellcolor[HTML]{FFFC9E}55 & \cellcolor[HTML]{FFFC9E}500 \\ \bottomrule
\end{tabular}
\end{table}
Die Randhäufigkeiten sind gelb markiert. 
\begin{enumerate}[label=\arabic* \bfseries Schritt:]
\item Signifikanzzahl 0.01 festgelegt. \\
Freiheitsgrade sind: 3 Qualitätsstufen $3-1=2$ und 4 Zweigwerke $4-1=3$. Multipliziert ergibt das, dass es 6 Freiheitsgrade gibt.
\item Der Testwert $y$ berechnet sich auch hier als Summe der quadrierten Differenzen zwischen tatsächlicher und erwarteter Häufigkeit jeweils dividiert durch die erwartete Häufigkeit. Die erwartete Häufigkeit ermittelt man durch die Randverteilung:
\begin{equation*}
\ra{1.5}
\begin{array}{rcl}
 \frac{h_{32}^{th}}{100}&=&\frac{110}{500}  \\
 h_{32}^{th}&=&\frac{110\cdot100}{500} \\
 h_{32}^{th}&=&22
\end{array}
\end{equation*}
Die Zahlen werden durch die Randhäufigkeiten in der Tabelle entnommen. $500$ ist die Summe aller Randhäufigkeiten, $100$ die Summe aus der Zeile 3, $110$ die Summe aus der Spalte 2, daher auch $h_{32}^{th}$
\item Daraus ergibt sich, mit Vorlage aus \autoref{eq:testverfahren:verteilung1}:
\begin{equation*}
 y=\frac{(125-107.2)^2}{107.2} + \frac{(25-35.2)^2}{35.2}+ \ldots + \frac{(20-14.85)^2}{14.85} = 20.72
\end{equation*}
Daraus ergibt sich eine Tabelle mit neuen Werten.
\begin{table}[htb]
\centering
\begin{tabular}{@{}lllll@{}} \toprule
\multicolumn{1}{c}{\multirow{2}{*}{Zweigwerk}} & \multicolumn{4}{c}{Qualitätsstufe} \\
\multicolumn{1}{c}{} & 1 & 2 & 3 & Summe \\ \midrule
1 & 107.2 & 35.2 & 17.6 & 160 \\
2 & 70.35 & 23.1 & 11.55 & 105 \\
3 & 67 & 22 & 11 & 100 \\
4 & 90.45 & 29.7 & 14.85 & 135 \\
Summe & 335 & 110 & 55 & 500 \\ \bottomrule
\end{tabular}
\end{table}
\end{enumerate}
$C_0$ aus der Tabelle ist $15.086 < 20.72$. Die Hypothese wird abgelehnt, die Ergebnisse der Zweigwerke sind nicht unabhängig.
\subsection{Zusammenhang zwischen zwei Merkmalen}
%------------------------------------------------------------------------------------
\subsubsection{Regressionsanalyse}
\paragraph{Zusammehang zwischen Merkmalen}
$n_w = 1500$ weibliche und $n_m = 500$ männliche Kunden haben alle einen Artikel gekauf, der in verschiedenen Farben verfügbar ist:
\begin{table}[ht]
\centering
\begin{tabular}{@{}lllll@{}}
\toprule
Farbe & Blau & Grün & Rot & Summe \\  \midrule 
Häufigkeit & 1000 & 600 & 400 & 2000 \\ \bottomrule
\end{tabular}
\end{table}
Wenn das Einkaufsverhalten unabhängig ist, so müssten die relativen Häufigkeiten bezogen auf Männer und Frauen gleich sein, oder mindestens sehr nahe beieinander liegen(siehe \autoref{eq:zusammenhang:1}).
\begin{table}[ht]
\centering
\begin{tabular}{@{}cllll@{}}
\toprule
\multirow{2}{*}{Geschlecht} & \multicolumn{4}{c}{Farbe} \\
 & Blau & Grün & Rot & Summe \\ \midrule
Weiblich & $(1000\cdot1500)\div2000$ & $(600\cdot1500)\div2000$ & $(400\cdot1500)\div2000$ & 1500 \\
Männlich & $(1000\cdot150)\div2000$ & $(1000\cdot150)\div2000$ & $(1000\cdot150)\div2000$ & 500 \\
\multicolumn{1}{l}{Summe} & \multicolumn{1}{l}{1000} & 600 & 400 & 2000 \\ \bottomrule
\end{tabular}
\end{table}\\
\smallskip
Wenn man nun die Resultate mit einer Tabelle aus, beispielsweise dem Vorjahr, vergleicht, und die Werte über "Pi-mal-Daumen" passen, sind die Merkmale eher voneinander abhängig.
\pagebreak
\paragraph{Minimiere die Summenformel}
\begin{equation*}
\sum_i^n(y_i - a - bx_i)^2
\end{equation*}
\begin{enumerate}
\item Partielles Ableiten des Ausdrucks nach $a$ und nach $b$
\item Nullsetzen der beiden partiellen Ableitungen
\item Auflösen der beiden Gleichungen nach $a$ und $b$
\end{enumerate}
\begin{equation*}
\ra{1.7}
\begin{array}{rcl}
a_1 &=& \overline{y}+b_1\cdot\overline{x}\\
b_1 &=& \displaystyle\frac{\sum_{i=1}^n x_i y_i - n\cdot\overline{x}\cdot\overline{y}}{\sum_{i=1}^n x_i^2 - n\overline{x}^2}
\end{array}
\end{equation*}
\subsubsection{Lineare Regression}
\paragraph{Beispiel} 12 Studenten gingen im letzten Semester neben dem Studium einer Erwerbstätigkeit nach. In der nachstehenden Tabelle sind für die 12 Studenten A bis L der zeitliche Aufwand (Std/Woche) für die Erwerbstätigkeit X und der zeitzliche Aufwand (Std./Woche) für das Studium Y angegeben
\smallskip
\begin{table}[H]
\ra{1.2}
\centering
\begin{tabular}{@{}lllllllllllll@{}}
\toprule
Student & A & B & C & D & E & F & G & H & I & J & K & L \\ \midrule
Erwerbstätigkeit & 1 & 2 & 2 & 3 & 3 & 4 & 5 & 6 & 8 & 12 & 15 & 33 \\
Studium & 39 & 37 & 36 & 40 & 36 & 37 & 34 & 36 & 33 & 33 & 32 & 27 \\ \bottomrule
\end{tabular}
\end{table}
Ein Student, der für die Bestreitung seines Lebensunterhalts 6 Stunden pro Woche erwärbstätig sein muss, will anhand der vorliegenden Daten ermitteln, wieviel Zeit er für sein Studium aufbringen kann. Die Berechnung erfolgt mit \autoref{eq:regression:4-1} und \autoref{eq:regression:4-2}.
\begin{table}[H]
\ra{1.2}
\centering
\begin{tabular}{@{}llllll@{}}
\toprule
Student & $x_i$ & $y_i$ & $x_i y_i$ & $x_i^2$ & $y_i^2$ \\ \midrule
A & 1 & 39 & 39 & 1 & 1521 \\
B & 2 & 37 & 74 & 4 & 1369 \\
C & 2 & 36 & 72 & 4 & 1296 \\
D & 3 & 40 & 120 & 9 & 1600 \\
E & 3 & 36 & 108 & 9 & 1296 \\
F & 4 & 37 & 148 & 16 & 1369 \\
G & 5 & 34 & 170 & 25 & 1156 \\
H & 6 & 36 & 216 & 36 & 1296 \\
I & 8 & 33 & 264 & 64 & 1089 \\
J & 12 & 33 & 396 & 144 & 1089 \\
K & 15 & 32 & 480 & 225 & 1024 \\
L & 23 & 27 & 621 & 529 & 729 \\
Summe & 84 & 420 & 2708 & 1066 & 14834 \\ \bottomrule
\end{tabular}
\end{table}
Wir können nun mit den oben erwähnten Gleichungen die Werte berechnen.
\begin{equation*}
\begin{array}{rcl}
b_1 &=& \frac{2708-12\cdot \frac{84}{12}\cdot \frac{420}{12}}{1066-12\cdot\left(\frac{84}{12}\right)^2} \\
&=& -\frac{116}{239} \\
&=& -0.49\\
a_1 &=&\frac{420}{12}-(-\frac{116}{239})\cdot \frac{84}{12} \\
&=& \frac{9177}{239} \\
&=& 38.43 
\end{array}
\end{equation*}
Daraus folgt zu allgemeinen Berechnung
\begin{equation*}
y=38.43 - 0.49 \cdot x \text{ mit } y(6) = 35.5\text{h}
\end{equation*}